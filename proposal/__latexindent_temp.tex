\documentclass[12pt]{extarticle}
\usepackage[utf8]{inputenc}
\usepackage{cite}
\usepackage{hyperref}

\title{Predicting Whether the Meaning of a Text is Positive or Negative}
\author{Nikolaos Gounakis, AM: 1254}
\date{December 2021}

\begin{document}

\maketitle

\section{Intro}
In applications like ChatBots or Video Games it is useful to understand
various concepts of text. One of them is the ability to understand if a piece of 
text provides a negative or a positive meaning. That is a binary classification 
problem and we propose a solution of constructing a ML model trained in 
Amazon Fine Food Reviews Dataset \cite{10.1145/2488388.2488466}. 

\section{Dataset}
This dataset consists of reviews of fine foods from amazon. 
The data span a period of more than 10 years, including all ~500,000 reviews up to October 2012. Reviews include product and user information, ratings, and a plain text review. 
It also includes reviews from all other Amazon categories.

The goal here is to determine wether a review is positive or negative.
\url{https://www.kaggle.com/snap/amazon-fine-food-reviews}

\section{Methodology}

There are several techniques to extract features from text but we will use 
and compare methods that are included in scikit-learn, such as the \textbf{Bag of Words} and
\textbf{TF-IDF}. We will use these methods to transform the text in to a feature
vector to provide as input to the algorithms.
\url{https://scikit-learn.org/stable/modules/feature_extraction.html}

Then we will train and evaluate models comparing the following algorithms for 
binary classification \textbf{KNN}, \textbf{SVM}, \textbf{Linear Regression} and 
\textbf{Naive Bayes}, \textbf{Random Forest}.

For each algorithm and feature extraction combination we will compute the following metrics:
\textbf{Accuracy}, \text{AUC} and

\bibliographystyle{plain}
\bibliography{bib/proposal}


\end{document}